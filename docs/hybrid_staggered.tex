%% 
%%  http://wso.williams.edu/wiki/index.php/LaTeX_Problem_Set_Template
%%

%%%%%%%%%%%%%%%%%%%%%%%%%%%%%%%%%%%%%%%%%%%%%%%
%%%This is a science homework template. Modify the preamble to suit
%%%your needs. 
%The junk text is   there for you to immediately see how the
%headers/footers look at first 
%typesetting.


\documentclass[12pt]{article}

%AMS-TeX packages
\usepackage{amssymb,amsmath,amsthm} 
%geometry (sets margin) and other useful packages
\usepackage[margin=1.25in]{geometry}
\usepackage{graphicx,ctable,booktabs}
\usepackage{url}

%    
\begin{document}

\title{Interpolating operators for hybrid mesons using staggered fermions }
%%\author{Craig McNeile}
\date{}

\maketitle


\section{Introduction}

See the review by Meyer about hybrid mesons~\cite{Meyer:2015eta}.

\section{Operators with staggered fermioms}

An operator to create a $1^{-+}$ state using staggered fermions
was first written down by the MILC collaboration~\cite{Bernard:2003jd}.

In the continuum~\cite{Lacock:1996vy,Bernard:1997ib},
the $1^{-+}$ operator can be
defined via $\rho \times B$,
where $\rho$ is the interpolating
operator for the vector meson and 
and the gluon field strength tensor.


\begin{figure}[htb]
\setlength{\unitlength}{0.5in}
\begin{picture}(6.0,2.3)(-3.0,-1.15)	%origin at center

% Four plaquettes making up F_mu_nu
\thicklines
\put(0.2,0.1){\vector(1,0){0.8}}
\put(1.0,0.1){\vector(0,1){0.9}}
\put(1.0,1.0){\vector(-1,0){0.9}}
\put(0.1,1.0){\vector(0,-1){0.8}}

\put(-0.1,0.2){\vector(0,1){0.8}}
\put(-0.1,1.0){\vector(-1,0){0.9}}
\put(-1.0,1.0){\vector(0,-1){0.9}}
\put(-1.0,0.1){\vector(1,0){0.8}}

\put(-0.2,-0.1){\vector(-1,0){0.8}}
\put(-1.0,-0.1){\vector(0,-1){0.9}}
\put(-1.0,-1.0){\vector(1,0){0.9}}
\put(-0.1,-1.0){\vector(0,1){0.8}}

\put(0.1,-0.2){\vector(0,-1){0.8}}
\put(0.1,-1.0){\vector(1,0){0.9}}
\put(1.0,-1.0){\vector(0,1){0.9}}
\put(1.0,-0.1){\vector(-1,0){0.8}}

% staples plus link making smeared link
\put(1.9,0.0){\vector(1,0){0.9}}
\put(3.0,-0.1){=}
\thinlines
\put(3.4,0.0){\vector(1,0){0.9}}
\put(3.4,0.1){\vector(0,1){0.9}}
\put(3.4,1.0){\vector(1,0){0.9}}
\put(4.3,1.0){\vector(0,-1){0.9}}
\put(3.4,-0.1){\vector(0,-1){0.9}}
\put(3.4,-1.0){\vector(1,0){0.9}}
\put(4.3,-1.0){\vector(0,1){0.9}}

\end{picture}
\caption{
``pointlike'' construction of $F_{\mu\nu}$. Each open loop represents
the product of the links, minus the adjoint of the product.
Each of these links may actually be a ``smeared'' link, as illustrated
on the right side of the figure.
}
\label{FIELDFIG}
\end{figure}

An explicit hybrid operaor in the continuum is
\begin{equation}
1^{-+}_x = \rho_y B_z - \rho_z  B_y
\end{equation}
where

\begin{equation}
\rho_i = \overline{\psi} \gamma_i \psi
\end{equation}

An early discussion of the group theory of 
the hybrid mesons on the lattice is by
Lacock et al.~\cite{Lacock:1996vy}.

There are a few additional issues with representing using the 
hybrid operator with staggered fermions. In this formalism
the $\gamma$ matrices are replaced by phases.
The $1^{-+}$ operator is $H_i$
\begin{equation}
H_i = \epsilon_{ijk} \overline{\psi}^a (\gamma_j \otimes 1) \psi^b
B_k^{ab}  \label{eq:hybStagg}
\end{equation}
where $i$, $j$, and $k$ are spatial indices and $a$ and $b$ 
are colour indices. The field strength $B_k^{ab}$ is
defined by gauge links.


The MILC collaboration used a taste singlet $\rho$ operator
$(\gamma_x \times  1 )$
The taste singlet staggered operator 
for the vector operator, which couples to the $\rho$ meson, is:
%%%
\begin{equation}
\rho_k = \overline{\chi}  \eta_k D_k \chi
\end{equation}
%%
where $\chi$ and $\overline{\chi}$ are quark and anti-quark fields.
The $\eta$ are the standard staggered phases.
See Gupta~\cite{Gupta:1997nd} for a review.
\begin{equation}
\eta_{x,\mu} = -1^{\sum_{\nu < \mu} x_\nu}
\end{equation}


\begin{equation}
D_\mu \chi(x) = \frac{1}{2}
[ U_\mu^\dagger(x - \hat{\mu}) \chi( x - \hat{\mu}) + U_{\mu} (x)
  \chi( x + \hat{\mu}) ]
\end{equation}


\subsection{Implementation in the MILC code}

The hybrid correlator is constructed below.
\begin{equation}
c(t)_i^{hyb}  = \sum_{x} \langle H^{\dagger}_i (x,t)  H(0,0)_i
\rangle
\end{equation}

The latest version of the MILC code uses a flexible system to create
meson correlators. Essentially a hybrid quark source is created using
the operator in equation~\ref{eq:hybStagg}. A quark propagator is then 
computed from the $H_i$ source. The $H_i$ source is then applied at
the
sink. The quark propagator with $H_i$  at the source and sink 
is then combined with a quark propagator
generated from a point source to form the hybrid meson correlator.


\subsection{Smearing and sources}

In the orginal paper by the MILC collaboration~\cite{Bernard:2003jd}, 
the field strength tensor was smeared by APE smearing.
The number of iterations was 32 and the relative weight was 0.25.


\bibliographystyle{h-physrev5}
\bibliography{hybrid}


\end{document}
